\section{Summary}
\label{sec:summary}

The production of Monte Carlo (MC) samples of adequate size is often a challenge in modern high-energy physics experiments,
due to the computing resources required to produce and store such samples.
This is particularly true for experiments at the CERN Large Hadron Collider (LHC),
firstly because of the large proton-proton scattering cross section and secondly because of the large luminosity delivered by the LHC.
In this paper we have presented a procedure, which reduces the statistical uncertainties arising from the limited size of MC samples used in physics analyses,
in case the regions covered by different MC samples overlap in phase-space (PS).
Our formalism is general enough to be applied in various different cases.
Different examples for applying the formalism were given in this paper.
Of particular interest is the case of searches for new physics,
where potential signals are typically expected to be small and often similar in size to the statistical uncertainties on background contributions, 
which arise from established Standard Model processes.
In this case, the statistical uncertainties on the background contributions can often be significantly reduced 
by dividing the PS into multiple regions, producing separate MC samples for each region,
and accounting for the overlap of different samples in PS by applying the weights computed as detailed in this paper to the simulated events.
We refer to this procedure as ``stitching''.
