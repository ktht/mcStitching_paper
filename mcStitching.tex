\def\verPreprint{1}
\def\verPAPER{2}
\def\ver{1}

\ifx\ver\verPreprint
\documentclass[a4paper,english,11pt]{article}
\usepackage[bindingoffset=0.5cm,left=2.5cm,right=2.5cm,top=2.5cm,bottom=2.5cm,footskip=1.0cm]{geometry}
\usepackage{lineno,hepnames,bm,multirow,amssymb,authblk,graphicx,newclude,xspace,hyperref,rotating}
\fi
\ifx\ver\verPAPER
\documentclass[1p]{elsarticle}
\usepackage{lineno,hyperref,hepnames,bm,multirow,amssymb,xspace,rotating}
\fi

%\modulolinenumbers[5]

%%\journal{Journal of \LaTeX\ Templates}

%%%%%%%%%%%%%%%%%%%%%%%
%% Elsevier bibliography styles
%%%%%%%%%%%%%%%%%%%%%%%
%% To change the style, put a % in front of the second line of the current style and
%% remove the % from the second line of the style you would like to use.
%%%%%%%%%%%%%%%%%%%%%%%

%% Numbered
%\bibliographystyle{model1-num-names}

%% Numbered without titles
%\bibliographystyle{model1a-num-names}

%% Harvard
%\bibliographystyle{model2-names.bst}\biboptions{authoryear}

%% Vancouver numbered
%\usepackage{numcompress}\bibliographystyle{model3-num-names}

%% Vancouver name/year
%\usepackage{numcompress}\bibliographystyle{model4-names}\biboptions{authoryear}

%% APA style
%\bibliographystyle{model5-names}\biboptions{authoryear}

%% AMA style
%\usepackage{numcompress}\bibliographystyle{model6-num-names}

%% `Elsevier LaTeX' style
\bibliographystyle{elsarticle-num}
%%%%%%%%%%%%%%%%%%%%%%%

%%%%%%%%%%%%%%%%%%%%%%%
%% Custom latex macros
%%%%%%%%%%%%%%%%%%%%%%%

\newcommand{\PZggx}{\ensuremath{\PZ/\Pgamma^{*}}\xspace}
\newcommand{\pT}{\ensuremath{p_{\textrm{T}}}\xspace}
\newcommand{\pThat}{\ensuremath{\hat{p}_{\textrm{T}}}\xspace}
%\newcommand{\kT}{\ensuremath{k_{\textrm{T}}}\xspace}
\newcommand{\kt}{\ensuremath{k_{\textrm{t}}}\xspace}
\newcommand{\HT}{\ensuremath{H_{\mathrm{T}}}\xspace}
\newcommand{\GeV}{\ensuremath{\textrm{GeV}}\xspace}
\newcommand{\TeV}{\ensuremath{\textrm{TeV}}\xspace}
\newcommand{\data}{\ensuremath{\textrm{data}}\xspace}
\newcommand{\mc}{\ensuremath{\textrm{mc}}\xspace}
\newcommand{\incl}{\ensuremath{\textrm{incl}}\xspace}
\newcommand{\jet}{\ensuremath{\textrm{j}}\xspace}
\newcommand{\pileup}{\ensuremath{\textrm{pu}}\xspace}
\newcommand{\Born}{\ensuremath{\textrm{born}}\xspace}
\newcommand{\ME}{\ensuremath{\textrm{me}}\xspace}
\newcommand{\MGvATNLO}{\textsc{MadGraph5}\_aMC@NLO\xspace}
\newcommand{\PYTHIA}{\textsc{Pythia}\xspace}
\newcommand{\cf}{cf.\xspace}
\newcommand{\ie}{i.e.\xspace}
\newcommand{\eg}{e.g.\xspace}
\def\TReg{\textsuperscript{\textregistered}}
\usepackage{array}
\newcolumntype{C}[1]{>{\centering\arraybackslash}p{#1}}
%%%%%%%%%%%%%%%%%%%%%%%

\begin{document}

\ifx\ver\verPAPER
\begin{frontmatter}
\fi

\title{Stitching Monte Carlo samples}

%% Group authors per affiliation:

\ifx\ver\verPreprint
\author[1]{Karl Ehat\"aht}
\author[1]{Christian Veelken}
\affil[1]{National Institute for Chemical Physics and Biophysics, 10143 Tallinn, Estonia}
\fi
\ifx\ver\verPAPER
\author[tallinn]{Karl Ehat\"aht}
\ead{karl.ehataht@cern.ch}
\author[tallinn]{Christian Veelken}
\ead{christian.veelken@cern.ch}
\address[tallinn]{National Institute for Chemical Physics and Biophysics, 10143 Tallinn, Estonia}
\fi

\ifx\ver\verPreprint
\maketitle
\fi

\begin{abstract}
Monte Carlo (MC) simulations are extensively used for various purposes in modern high-energy physics experiments.
Precise measurements or searches for new physics often require the collection of vast amounts of data.
It is often difficult to produce equally large MC samples, due to the computing resources that would be required to produce and store such samples.
One solution often employed when analyzing data recorded by high-energy particle experiments 
is to divide the phase-space of particle interactions into multiple regions 
and to adapt the size of MC samples that are produced for each region to the needs of physics analyses that are performed in these regions.
In this paper we describe a procedure for making optimal use of the available MC samples, 
in terms of reducing the statistical uncertainties that arise in physics analyses from the limited size of MC samples,
in case the regions covered by different MC samples overlap in phase-space.
Different examples are given for applying the procedure to proton-proton collisions at the CERN Large Hadron Collider.
\end{abstract}

\ifx\ver\verPAPER
\end{frontmatter}
\fi

\clearpage

%\linenumbers

%\begingroup
%\let\clearpage\relax
\include*{introduction}
\include*{stitching_weights}
\include*{examples}
\include*{summary}

\appendix
\include*{appendix}

\bibliography{mcStitching}
%\endgroup

\end{document}
