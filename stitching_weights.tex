\section{Computation of stitching weights}
\label{sec:stitching_weights}

As explained in the introduction,
the production of MC samples by contemporary high-energy physics experiments often utilises production schemes
which divide the PS into multiple regions and adapt the size of MC samples that are produced in each region 
according to the needs of different physics analyses.
When using these MC samples for physics analyses,
weights need to be applied to the simulated events, in order to obtain background estimates that are unbiased.
More specifically, the weights need to be chosen such that the weighted sum of simulated events in each region $i$ of PS 
matches the SM prediction in that region:
\begin{equation}
\sum_{j} \, N_{j} \, p_{j}^{i} \, w_{j}^{i} = L \, \sigma_{i} \, ,
\label{eq:one}
\end{equation}
where the symbol $L$ corresponds to the integrated luminosity of the analysed dataset
and $\sigma_{i}$ denotes the cross section predicted by the SM in the PS region $i$.
The sum on the left-hand-side extends over the MC samples $j$,
with $N_{j}$ denoting the total number of simulated events in the $j$-th sample.
The symbol $p_{j}^{i}$ corresponds to the probability for an event in MC sample $j$ to fall into PS region $i$,
and $w_{j}^{i}$ denotes the weight applied to events from the $j$-th sample that fall into this PS region.
Eq.~(\ref{eq:one}) holds separately for each background process under study.
In principle, the same equation applies to signal MC samples. 
The case of signal MC samples is less relevant, however,
as signal cross sections are typically significantly smaller that those of background processes 
and the production of signal MC samples of sufficient size is rarely a problem in practice.

One can show that the lowest statistical uncertainty on the background estimate is achieved 
if the weights $w_{j}^{i}$ in Eq.~(\ref{eq:one}) only depend on the PS region $i$,
that is, if all simulated events that fall into the same region of PS have the same weight,
regardless of which MC sample contains these events.
We hence use weights that depend only in the PS region $i$ and not on the MC sample $j$ 
and refer to these weights using the notation $w^{i}$.

It is useful to express the cross section $\sigma_{i}$ as the product of an ``inclusive'' cross section $\sigma_{\incl}$,
which refers to the whole PS, and the probability $p^{i}$ that an event generated in the inclusive PS falls into region $i$:
\begin{equation*}
\sigma_{i} = \sigma_{\incl} \, p^{i} \, .
\label{eq:two}
\end{equation*}
Inserting this relation into Eq.~(\ref{eq:one}) and solving for the weight $w^{i}$, one obtains:
\begin{equation}
w^{i} = \frac{L \, \sigma_{\incl} \, p^{i}}{\sum_{j} \, N_{j} \, p_{j}^{i}} \, .
\label{eq:master}
\end{equation}

In practical applications of the stitching procedure,
one often faces the case where one MC sample, produced for general physics analyses, covers the whole PS
and where this ``inclusive'' sample is complemented by additional samples, which cover the regions of PS most relevant for searches for new physics.
In this case, Eq.~(\ref{eq:master}) can be rewritten in the form:
\begin{equation}
w^{i} = \frac{L \, \sigma_{\incl}}{N_{\incl}} \cdot \frac{N_{\incl} \, p^{i}}{N_{\incl} \, p^{i} + \sum_{j} \, N_{j} \, p_{j}^{i}} \, ,
\label{eq:incl}
\end{equation}
where $N_{\incl}$ refers to the size of the inclusive sample.
Here, the sum over $j$ extends to the additional samples that cover distinct regions of PS.
We will refer to these samples as ``exclusive'' samples.
The two factors in Eq.~(\ref{eq:incl}) may be interpreted in the following way:
The first factor corresponds to the weights that one would apply to the events in PS region $i$ 
in case no exclusive samples are available and the background estimate in PS region $i$ is obtained using the inclusive sample only.
The production of exclusive samples increases the number of simulated events falling into the PS region $i$ 
from $N_{\incl} \, p^{i}$ to $N_{\incl} \, p^{i} + \sum_{j} \, N_{j} \, p_{j}^{i}}$ 
and reduces the weights that are applied to simulated events falling into this region by the reciprocal of this increase,
given by the second factor in Eq.~(\ref{eq:incl}).
This in turn reduces the statistical uncertainty on the background estimate in the PS region $i$ by a factor 
$\sqrt{\frac{N_{\incl} \, p^{i}}{N_{\incl} \, p^{i} + \sum_{j} \, N_{j} \, p_{j}^{i}}}$.
